\documentclass[11pt,a4paper]{article}
\usepackage[utf8]{inputenc}
\usepackage{amsmath,amssymb,amsthm}
\usepackage[margin=1in]{geometry}
\usepackage{enumitem}

\newtheorem{theorem}{Theorem}
\newtheorem{definition}{Definition}
\newtheorem{example}{Example}

\title{Reading Report: GIG Random Matrices and a Yang--Baxter Extension of the Matsumoto--Yor Property}
\author{arXiv:2602.12713 --- G.~Letac, M.~Piccioni, J.~Weso{\l}owski}
\date{February 2026}

\begin{document}
\maketitle

%% ---------------------------------------------------------------
\section{Core Question}
%% ---------------------------------------------------------------

Given two independent random matrices $X,Y$ taking values in the cone of symmetric positive-definite (SPD) matrices, a certain rational map $\varphi^{(\alpha,\beta)}$ produces a new pair $(U,V)$.
The paper asks: \emph{if $(U,V)$ are again independent, must $X$ and $Y$ follow matrix-variate Generalized Inverse Gaussian (MGIG) distributions?}
A secondary question is whether the map $\varphi^{(\alpha,\beta)}$ satisfies the parametric Yang--Baxter equation on the SPD cone.

%% ---------------------------------------------------------------
\section{Exact Main Statements}
%% ---------------------------------------------------------------

\paragraph{Setup and notation.}
Let $r\ge 1$ be fixed.  Write $\Omega_+$ for the cone of $r\times r$ real symmetric positive-definite matrices, equipped with inner product $\langle a,b\rangle = \operatorname{tr}(ab)$.

\begin{definition}[Matrix GIG distribution]
A random matrix $W\in\Omega_+$ has distribution $\mathrm{MGIG}(\lambda, a, b)$ with $\lambda\in\mathbb{R}$, $a,b\in\Omega_+$, if its density on $\Omega_+$ is
\[
  f(x) \;\propto\; (\det x)^{\lambda - \frac{r+1}{2}}\,
  \exp\!\bigl(-\langle a,x\rangle - \langle b,x^{-1}\rangle\bigr).
\]
\end{definition}

\begin{definition}[The map $\varphi^{(\alpha,\beta)}$]
For $\alpha,\beta\ge 0$, define $\varphi^{(\alpha,\beta)}\colon \Omega_+^2\to\Omega_+^2$ by
\[
  \varphi^{(\alpha,\beta)}(x,y)
  = \Bigl(y\,(I+\alpha xy)^{-1}(I+\beta xy),\;
         x\,(I+\beta yx)^{-1}(I+\alpha yx)\Bigr).
\]
\end{definition}

\begin{theorem}[Direct result --- Letac--Weso{\l}owski 2024]\label{thm:direct}
If $X\sim\mathrm{MGIG}(\lambda,\alpha a,b)$ and $Y\sim\mathrm{MGIG}(\lambda,\beta b,a)$ are independent, and $(U,V)=\varphi^{(\alpha,\beta)}(X,Y)$, then $U$ and $V$ are independent with
\[
  U\sim\mathrm{MGIG}(\lambda,\alpha b,a),\qquad
  V\sim\mathrm{MGIG}(\lambda,\beta a,b).
\]
\end{theorem}

\begin{theorem}[Characterization --- Theorem 1.2 of the paper]\label{thm:char}
Let $\alpha\neq\beta$ and let $X,Y$ be independent $\Omega_+$-valued random matrices with strictly positive $C^2$ densities on $\Omega_+$.  If $(U,V)=\varphi^{(\alpha,\beta)}(X,Y)$ are independent, then there exist $\lambda\in\mathbb{R}$ and $a,b\in\Omega_+$ such that
\[
  X\sim\mathrm{MGIG}(\lambda,\alpha a,b),\quad
  Y\sim\mathrm{MGIG}(\lambda,\beta b,a),\quad
  U\sim\mathrm{MGIG}(\lambda,\alpha b,a),\quad
  V\sim\mathrm{MGIG}(\lambda,\beta a,b).
\]
\end{theorem}

\begin{theorem}[Yang--Baxter property --- Theorem 1.3]\label{thm:yb}
The map $\varphi^{(\cdot,\cdot)}$ is a parametric Yang--Baxter map on $\Omega_+$, i.e.\ the equation
\[
  F^{(\alpha,\beta)}_{12}\circ F^{(\alpha,\gamma)}_{13}\circ F^{(\beta,\gamma)}_{23}
  = F^{(\beta,\gamma)}_{23}\circ F^{(\alpha,\gamma)}_{13}\circ F^{(\alpha,\beta)}_{12}
\]
holds on $\Omega_+^3$, where $F^{(\alpha,\beta)}_{12}(x,y,z)=(\varphi_1^{(\alpha,\beta)}(x,y),\varphi_2^{(\alpha,\beta)}(x,y),z)$, etc.
\end{theorem}

%% ---------------------------------------------------------------
\section{A Concrete Example: the Scalar Case}
%% ---------------------------------------------------------------

\begin{example}[Matsumoto--Yor property, $r=1$, $\alpha=1$, $\beta=0$]
Take $r=1$ so that $\Omega_+=\mathbb{R}_{>0}$.  A scalar $\mathrm{GIG}(-\lambda,\gamma_1,\gamma_2)$ has density $f(x)\propto x^{-\lambda-1}e^{-\gamma_1 x - \gamma_2/x}$ on $(0,\infty)$.

Let $X\sim\mathrm{GIG}(-\lambda,\gamma_1,\gamma_2)$ and $Y\sim\mathrm{Gamma}(\lambda,\gamma_1)$ be independent.  Set
\[
  U = \frac{1}{X+Y},\qquad V = \frac{1}{X} - \frac{1}{X+Y}.
\]
Then $U\sim\mathrm{GIG}(-\lambda,\gamma_2,\gamma_1)$ and $V\sim\mathrm{Gamma}(\lambda,\gamma_2)$, and $U\perp V$.
This is the \emph{classical Matsumoto--Yor property} (2001), originally discovered via exponential Brownian motion.

The general $(\alpha,\beta)$ case (Croydon--Sasada 2020) replaces the Gamma with another GIG and connects to discrete KdV integrable systems.
\end{example}

%% ---------------------------------------------------------------
\section{Intuition and Setup}
%% ---------------------------------------------------------------

The Matsumoto--Yor (MY) property is a rare ``independence preservation'' phenomenon: a nonlinear, non-affine map sends a pair of independent random variables to another independent pair, and the distributions involved are precisely GIG and Gamma.  Such phenomena are scarce and typically tied to deep algebraic structure.

Croydon and Sasada (2020) discovered that the MY property is a special case ($\alpha=1,\beta=0$) of a broader family of independence-preserving maps $H^{(\alpha,\beta)}_{III,B}$ arising from the \emph{discrete Korteweg--de Vries (KdV) equation}.  These maps are \emph{parametric Yang--Baxter maps}, meaning they satisfy the Yang--Baxter equation --- a consistency condition central to integrable systems.

Letac and Weso{\l}owski (2024) extended the map $H^{(\alpha,\beta)}_{III,B}$ from scalars to SPD matrices, proving the \emph{direct} independence-preserving result (Theorem~\ref{thm:direct}).  The present paper completes the picture:
\begin{itemize}[nosep]
  \item \textbf{Characterization (converse):} MGIG distributions are the \emph{only} laws for which $\varphi^{(\alpha,\beta)}$ preserves independence.
  \item \textbf{Yang--Baxter:} The matrix map $\varphi^{(\alpha,\beta)}$ inherits the YB property from its scalar ancestor.
\end{itemize}

%% ---------------------------------------------------------------
\section{Main Results (Bulleted)}
%% ---------------------------------------------------------------

\begin{itemize}
  \item \textbf{Characterization of MGIG by independence} (Theorem~\ref{thm:char}): Under $C^2$, strictly-positive density assumptions, independence of $(U,V)=\varphi^{(\alpha,\beta)}(X,Y)$ with $\alpha\neq\beta$ forces all four variables $X,Y,U,V$ to be MGIG with linked parameters.

  \item \textbf{Yang--Baxter equation on $\Omega_+$} (Theorem~\ref{thm:yb}): $\varphi^{(\alpha,\beta)}$ is a parametric YB map, extending the known scalar result to the non-commutative cone of SPD matrices.

  \item \textbf{Unification:} The classical MY property ($\alpha=1,\beta=0$), its matrix version, and the Croydon--Sasada KdV connection are all special cases of the framework.
\end{itemize}

%% ---------------------------------------------------------------
\section{Proof Sketch}
%% ---------------------------------------------------------------

\subsection*{Characterization (Theorem~\ref{thm:char})}

The proof proceeds in ten steps.  We outline the key ideas.

\begin{enumerate}[label=\textbf{Step \arabic*.}, leftmargin=2.5em]

\item \textbf{Reduction to a scalar functional equation.}
Independence of $(U,V)$ yields a functional equation for the log-densities:
\[
  g_U(u) + g_V(v) = g_X(x) + g_Y(y), \quad (u,v)=\psi(x,y).
\]
Restricting to the one-parameter subfamily $(x,y)=(\sigma w,\tau w)$ for fixed $w\in\Omega_+$ reduces this to a \emph{scalar} functional equation of the form
\[
  m_U\!\Bigl(\tfrac{\tau+\beta\sigma}{\tau+\alpha\sigma}\,\tfrac{1}{\tau}\Bigr)
  + m_V\!\Bigl(\tfrac{\tau+\beta\sigma}{\tau+\alpha\sigma}\,\tfrac{1}{\sigma}\Bigr)
  = m_X(\sigma)+m_Y(\tau).
\]

\item \textbf{Invoke the scalar characterization.}
Multiply the restricted log-densities by known scalar GIG densities to construct integrable densities satisfying the scalar version of the functional equation (equation~(1.6) in the paper).  By the \emph{known univariate characterization} (Letac--Weso{\l}owski 2024), these must be GIG.  This gives, for each $w$, a parametric form:
\[
  g_X(\tau w) = -l(w)\log\tau - \alpha\, a(w)\,\tau - \frac{b(w)}{\tau} + C_X(w),
\]
and analogously for $Y,U,V$, with functions $a,b,l,C_W$ depending on $w$.

\item \textbf{Scaling analysis.}
Replace $w$ by $\rho w$ and use the known scaling of each $h_W^{(w)}$ to deduce homogeneity:
\[
  a(\rho w) = \rho\, a(w),\quad b(\rho w) = \rho^{-1}b(w),\quad l(\rho w) = l(w).
\]

\item \textbf{Separation into individual equations.}
Set $\tau=1$, substitute the parametric forms back into the full matrix functional equation, and use the linear independence of $\{1,\rho,\rho^{-1},\log\rho\}$ to obtain \emph{separate} matrix functional equations for $a$, $b$, $l$, and $C_W$.

\item \textbf{Identify $a(w) = \operatorname{tr}[a\, w]$ for some $a\in\Omega_+$.}
Differentiate the equation for $a$ twice (with respect to $x$ and then $y$), specialize to $y=\rho x$, and use the factorization of a coefficient $C(\rho)=(\rho^2-\alpha\beta)/(\beta+\rho)^3$ (which is nonzero for suitable $\rho$) to conclude $a''(x)=0$.  Combined with homogeneity, $a(w)=\operatorname{tr}[a\,w]$.

\item \textbf{Identify $b(w) = \operatorname{tr}[b\, w^{-1}]$ for some $b\in\Omega_+$.}
Define $\tilde{b}(w)=b(w^{-1})$; then $\tilde{b}$ satisfies the same homogeneity and functional equation as $a$.  Hence $\tilde{b}(w)=\operatorname{tr}[b\,w]$, i.e.\ $b(w)=\operatorname{tr}[b\,w^{-1}]$.

\item \textbf{Identify $l\equiv \text{const}$.}
A similar double-differentiation argument yields a matrix ODE $-2l''(x)[h] = l'(x)hx^{-1} + x^{-1}h\,l'(x)$.  An asymptotic argument ($y\to\rho y$, $\rho\to\infty$) gives a trace identity, which together with the ODE forces $l'(x) = c\,x^{-1}$ for a scalar matrix $c$.  A commutation argument (using $cy = yc$ for all $y\in\Omega_+$) shows $c\propto I$, and homogeneity $l(\rho w)=l(w)$ then forces $c=0$, so $l$ is constant.

\item \textbf{Identify the constants $C_W$.}
Since $l$ is constant, the equations for $C_W$ reduce to the same structural form.  Separation of variables and a limiting argument ($\rho\to 0^+$) show each $F_W$ (a modified $C_W$) is constant.

\item \textbf{Assemble the distributions.}
Substituting $a(w)=\operatorname{tr}[aw]$, $b(w)=\operatorname{tr}[bw^{-1}]$, $l=\tilde\lambda$, and the constants into the log-densities recovers exactly the MGIG form for $X,Y,U,V$ with linked parameters.

\item \textbf{Integrability forces $a,b\in\Omega_+$.}
Since all four densities must integrate to~1 over $\Omega_+$, the parameter matrices $a,b$ must be strictly positive definite.
\end{enumerate}

\subsection*{Yang--Baxter Property (Theorem~\ref{thm:yb})}

The proof is a direct algebraic verification on $\Omega_+^3$.  One must show
\[
  (x_3,y_3,z_3) = (X_3,Y_3,Z_3)
\]
where both triples are obtained by composing the maps $F_{12},F_{13},F_{23}$ in the two different orders.  Each component ($x_3=X_3$, $z_3=Z_3$, $y_3=Y_3$) is verified separately:
\begin{itemize}[nosep]
  \item For $x_3=X_3$ and $z_3=Z_3$: expand using the identity $s(I+ts)=(I+st)s$ and its inverse version, then simplify.  The two cases are symmetric (swap roles of $x$ and $z$).
  \item For $y_3=Y_3$: the most involved calculation.  After substitution $u=yx$, $v=yz$, the equation reduces to verifying an identity involving $(I+\kappa s)^{-1}$ for $\kappa\in\{\alpha,\beta,\gamma\}$ and $s=u+v+\beta uv$.  A key step uses $\alpha=\gamma$ as trivial, then for $\alpha\neq\gamma$ factors the difference into two identities, both following from $s(I+\beta u)=(I+\beta v)^T s^T$-type commutation relations.
\end{itemize}

\end{document}
